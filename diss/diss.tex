\documentclass[12pt,twoside,notitlepage]{report}

\usepackage{a4}
\usepackage{verbatim}
\usepackage{algpseudocode}
\usepackage{algorithm}
\usepackage{listings}
\usepackage{minted}
\usepackage{tikz}
\usepackage{pgfplots}

\input{epsf}                            % to allow postscript inclusions
% On thor and CUS read top of file:
%     /opt/TeX/lib/texmf/tex/dvips/epsf.sty
% On CL machines read:
%     /usr/lib/tex/macros/dvips/epsf.tex



\raggedbottom                           % try to avoid widows and orphans
\sloppy
\clubpenalty1000%
\widowpenalty1000%

\addtolength{\oddsidemargin}{6mm}       % adjust margins
\addtolength{\evensidemargin}{-8mm}

\renewcommand{\baselinestretch}{1.1}    % adjust line spacing to make
                                        % more readable

\begin{document}

\bibliographystyle{plain}

\newcommand{\name}{Joseph Seaton}
\newcommand{\college}{Fitzwilliam College}
\newcommand{\ptitle}{Shader Compositor}

%%%%%%%%%%%%%%%%%%%%%%%%%%%%%%%%%%%%%%%%%%%%%%%%%%%%%%%%%%%%%%%%%%%%%%%%
% Title


\pagestyle{empty}

\hfill{\LARGE \bf \name}

\vspace*{60mm}
\begin{center}
\Huge
{\bf \ptitle} \\
\vspace*{5mm}
Part II Computer Science \\
\vspace*{5mm}
\college \\
\vspace*{5mm}
\today  % today's date
\end{center}

\cleardoublepage

%%%%%%%%%%%%%%%%%%%%%%%%%%%%%%%%%%%%%%%%%%%%%%%%%%%%%%%%%%%%%%%%%%%%%%%%%%%%%%
% Proforma, table of contents and list of figures

\setcounter{page}{1}
\pagenumbering{roman}
\pagestyle{plain}

\chapter*{Proforma}

{\large
\begin{tabular}{ll}
Name:               & \bf \name    \\
College:            & \bf \college \\
Project Title:      & \bf \ptitle  \\
Word Count:         & \bf 0\\
Project Originator: & Christian Richardt                    \\
Supervisor:         & Christian Richardt                    \\ 
\end{tabular}
}
% detex diss.tex | tr -cd '0-9A-Za-z $\tt\backslash$n' | wc -w
\stepcounter{footnote}


\section*{Original aims of the project}
To implement a novel user interface for the creation, testing and easy modification of pipelines of shaders. To automatically detect shader parameters and provide a simple interface to change them. To apply some optimisations to parts of this process.

\section*{Work completed}
The core of the project has been completed and works satisfactorily. Shaders can be created, and a pipeline of shaders can be specified using JavaScript, the result of which is shown in-browser. An interface for modifying shader parameters is correctly generated for [TODO: all] most types of parameter. A number of optimisations have been applied, including elimination of unnecessary shader recompilation and pipeline re-generation. [PENDING] The project is also now capable of reusing FBOs between pipeline changes where possible.

\section*{Special difficulties}
None.
 
\newpage
\section*{Declaration}

I, \name \ of \college, being a candidate for Part II of the Computer
Science Tripos, hereby declare
that this dissertation and the work described in it are my own work,
unaided except as may be specified below, and that the dissertation
does not contain material that has already been used to any substantial
extent for a comparable purpose.

\bigskip
\leftline{Signed}

\medskip
\leftline{Date }

\cleardoublepage

\tableofcontents

%%%%%%%%%%%%%%%%%%%%%%%%%%%%%%%%%%%%%%%%%%%%%%%%%%%%%%%%%%%%%%%%%%%%%%%
% now for the chapters

\cleardoublepage        % just to make sure before the page numbering
                        % is changed

\setcounter{page}{1}
\pagenumbering{arabic}
\pagestyle{headings}

\chapter{Introduction}
My project provides a novel user interface for easily compositing multiple shaders together, in which individual shader parameters can be easily modified and the results viewed immediately. I have successfully implemented the core of my project, including a number of proposed extensions.

\section{Motivation}
Working with OpenGL is usually undeniably messy. OpenGL is a verbose and complicated API. When developing shaders, developers tend not to be interested in writing large amounts of OpenGL API code just to tweak shader parameters or forward the texture output of one shader to the input of another. But this is exactly what developers have to do. Given that many graphics applications are written in compiled languages like C and C++, this significantly increases the length of a development cycle. Given that a large part of shader development constitutes small tweaks to (often subjectively) improve output, this is not ideal.

For students learning about shaders, OpenGL is often an intimidating and confusing mess that gets in the way of understanding a fundamentally quite simple concept.

This project aims to provide a way of developing shaders that is both useful for developers, and may be used as a tool for students learning about shaders for the first time.
\section{Brief introduction to shaders}
[TODO] Use of shader program vs shader is messy and technically incorrect throughout.
In their modern incarnation, OpenGL shaders are quite a simple concept to understand at a high level. A shader is a simple program -- `simple' referring to certain constraints we shall ignore for now -- written in a dataflow style. Such programs take in one item of data at a time, for example a vertex, and output another item of data -- the new vertex location, or a pixel value.
In a modern OpenGL implementation, there are two commonly used types of shader (and a few more which will not be discussed). These are vertex shaders, and fragment shaders. 
\subsection{A look at some OpenGL calls}
As discussed above, the use of shaders sounds simple. However, as we can see from [REF listing], a short excerpt of the OpenGL calls from an actual WebGL program [REF], the paraphanalia of mostly boilerplate OpenGL calls required for compiling shader programs, specifying parameters and so on can result in a very large codebase for even the simplest program.
\begin{minted}{javascript}
rawData = new Uint8Array(noisepixels);
texture_noise_l = gl.createTexture();
gl.bindTexture(gl.TEXTURE_2D, texture_noise_l);
gl.pixelStorei(gl.UNPACK_ALIGNMENT, 1);
gl.texImage2D(gl.TEXTURE_2D, 0, gl.RGBA, sizeX, sizeY, 0, gl.RGBA,\\ gl.UNSIGNED_BYTE, rawData);
gl.texParameteri(gl.TEXTURE_2D, gl.TEXTURE_MIN_FILTER, gl.LINEAR);
gl.texParameteri(gl.TEXTURE_2D, gl.TEXTURE_MAG_FILTER, gl.LINEAR);
\end{minted}
\cleardoublepage
\chapter{Preparation}

\section{Development environment}
\subsection*{WebGL}
WebGL is a very recently developed API allowing web-based applications access to OpenGL ES contexts, via JavaScript [REF]. It was decided that the use of web-based technologies would help lower the barrier for use of this project, and therefore shader development in general, since a user needs only visit the correct page using a modern webbrowser. Using WebGL has the added advantage of requiring less familiarisation given my prior experience with JavaScript. In [REF], we can see that WebGL uses essentially the same API as C does OpenGL ES, except 



\subsection*{Emacs}
Surprisingly for such an old editor, with the correct extensions Emacs is very well equipped for modern JavaScript development. While there are other editors available that could claim most or all of the features these extensions add, my previous experience with Emacs makes this a favourable choice. The extensions used add auto-completion, syntax checking, and an inline node.js (see below) console [REF http://blog.deadpansincerity.com/2011/05/setting-up-emacs-as-a-javascript-editing-environment-for-fun-and-profit/].

\subsection*{node.js}
node.js is a JavaScript platform built on top of Google's V8 JavaScript engine. It provides among other things, a command line interface, including a REPL (read-evaluate-print-loop). This provides a fast, light-weight way to test non-browser-bound code using traditional command line tools, can be easily interfaced with git and cron, and can easily write to files. All of this can be done without requiring messy extra code to interact with the webbrowser or server-side code to report back statistics.

\subsection*{Google Chrome}
WebGL naturally requires the use of a webbrowser. I have found for my particular Operating System / hardware combination, WebGL on Google Chrome is more stable. Since WebGL is a relatively recent development that has only recently gained browser support, WebGL can still be unstable and buggy in some circumstances even under Google Chrome. Google Chrome also includes a set of 'developer tools' including 

\subsection*{WebGL Inspector}
While the premise of the WebGL inspector tool - a Google Chrome extension that provides inline information on current WebGL contexts - is promising, I found the tool to be too unstable to be helpful in practice.

\section{Software development process}
Given my own lack of familiarity with the technologies used, and the somewhat experimental nature of WebGL, an iterative development process was deemed necessary.

I decided to use a combination of a test-driven and evolutionary software development model.

\section{Preparatory learning}
\subsection{Background reading}

\subsubsection{OpenGL and GLSL}
Prior to beginning this project, I had very little knowledge of OpenGL barring a small amount of personal experience writing toy programs in my spare time. As such I needed to gain a better understanding of OpenGL, and some knowledge of GLSL before I could begin this project. For this purpose I obtained a copy of the famous `Red Book' [REF]. I also read through the GLSL ES specification provided by the Khronos Group [REF], which discusses the syntax of GLSL in detail.

\subsubsection{JavaScript}
While, like many people I already have some experience of JavaScript programming from web development work, I thought it would be wise to refresh myself before embarking on a more complex project such as this one. I found particularly useful [REF http://eloquentjavascript.net] and [http://w3future.com/html/stories/hop.xml], both of which are unusual among texts on JavaScript in being willing to discuss the more sophisticated functional aspects of JavaScript, and their relation to JavaScript's own unusual prototype-based object system.


\section{Libraries}
JavaScript as provided by webbrowsers is usually quite 

\subsection{jQuery}
jQuery is a general utility library for JavaScript, providing many helper functions not provided by browsers, and also greatly simplified DOM (Document Object Model) manipulation. In \ref{jq} we see the difference in concision with and without jQuery for some simple common tasks.
\begin{minted}{javascript}
// jQuery

\$("a").clickfunction() {
  ...
})

// JavaScript

[].forEach.call(document.querySelectorAll("a"), function(el) {
  el.addEventListener("click", function() {
    ...
  });
});
\end{minted}

\subsection{Testing framework}
There are many unit testing frameworks available for JavaScript, of varying levels of complexity. Many of these frameworks are designed with very large projects with extensive tests in mind. For my project I considered these to be overly complicated to work with. Eventually I settled on QUnit, a very simple unit testing framework that is part of jQuery, and it's port to node.js (a fork of the original code).

\subsection{WebGL toolkit}
Given that OpenGL and therefore WebGL itself can be quite awkward to work with, I decided to use an existing library to abstract away some of the boilerplate code irrelevant to my project. However, there are currently many libraries available claiming to do exactly this, of varying degrees of completeness and varying levels of abstraction. Initially I settled on three.js [REF] on the basis of its popularity, and high rate of development. However it soon became apparent that in the level of abstraction provided by three.js actually made the task more difficult, given that interacting with FBOs was non-obvious. After some more searching I came across \textsc{glow} [REF], a toolkit specifically designed to make working with shaders simple, but otherwise providing little abstraction.

\subsection{GUI toolkit}
\subsubsection{Editor}
A core aim of the project was to provide good GLSL syntax highlighting. There currently exist many web-based editors offering some degree of syntax highlighting, but I chose to use CodeMirror [REF], given its active development, and the presence of a modular way to add support for highlighting new grammars. 

\subsubsection{jQuery UI}
jQuery UI is a User Interface library built on top of jQuery. It provides a number of common, basic widgets.

\section{Version control and backup strategy}
All project files, including the dissertation, were placed under git version control. Git was then configured to push to multiple remote repositories in different locations: an external harddrive, my PWF account, my SRCF account, my web hosting account, and GitHub. While this could be considered excessive, the use of SSH keys meant that post-setup, this required no extra effort on my part. Since I am conscious that git is capable of history rewriting, I also configured a regular cron job to make regular compressed backups of my current repository using git bundle.

\cleardoublepage
\chapter{Implementation}
The project consists of the following parts:
\subsection{Parser}
The parser extracts relevant parameters from a given shader, and their associated types.
\subsection{Pipeline specification}
The user-specified shader tree [TODO use shader tree rather than pipeline more] is extracted and converted into a usable format
\subsection{Parameter UI generation}
An appropriate user interface is generated based on the pipeline specification and the shader parameters extracted by the parser.
\subsection{Pipeline generation}
A list of shaders is generated from the shader tree such that the shaders can be rendered in sequence.
\subsection{Pipeline initialization}
Actual GLOW shader objects are created. FBOs are assigned to shaders as necessary.
\subsection{Rendering}
Render the actual pipeline, resulting in a preview image in the UI.
\subsection{User Interface}

\section{Parser}
The main job of the parser is to take a shader program, and provide an array of parameter names and their corresponding data types, which must be supplied to a shader program for it to run. The parser may also return information for the syntax highlighter. Since WebGL is based on a very specific version of OpenGL, OpenGL ES 2.0, the parser only needs to support GLSL ES as defined by [REF http://www.khronos.org/registry/gles/specs/2.0/GLSL_ES_Specification_1.0.17.pdf]. [REF http://www.khronos.org/registry/webgl/specs/latest] This simplifies the construction of the parser, since we need not worry about different GLSL versions.

Since GLSL's is (mostly) LALR, it was deemed sensible to use a parser generator rather than expend  effort writing a parser by hand. Initially I chose to use JS/CC [REF], since it seemed well-documented and quite popular. However, once I had used it to generate an appropriate parser for GLSL, it quickly became apparent that the parser was insufficiently fast to use for such a complicated grammar. In fact, I was unable to obtain any timing for the parser, as under Google Chrome, even with the simplest inputs it would fail to produce any output before Chrome itself decided to kill the process. While it may have been possible to obtain timing information, this was clearly too slow to be of any value. I then found Jison [REF], a JavaScript port of Bison. This was able to parse even quite complicated input in a reasonable amount of time, as is discussed in my evaluation.

\subsubsection{Grammar conversion}
The grammar for GLSL is provided for GLSL ES by [REF]. 

\subsubsection{Struct parsing}
The GLSL Specification [REF] allows a shader program to define and use new types of structures. These structures behave in essentially the same way as structs in C, subject to certain restrictions. Structure names and field names are subject to the same restrictions as normal identifiers. This makes parsing GLSL as parsed by e.g. Google Chrome technically context-sensitive. However, the parser used, Jison, only supports LALR grammars.

The grammar as provided in the specification introduces extra tokens for structure and field names, ignoring the context sensitivity. Therefore by way of an initial, straightforward implementation of struct parsing, I chose to make the parser simply append new structure/field names to the appropriate part of the matching logic of the lexer. While this will fail to parse shaders that use e.g. some field name as the name of a variable, 

\section{Pipeline specification}
The purpose of the pipeline specification stage is to take input from the user specifying the shaders to be used, and the connections between them - that is, when a shader uses the output FBO of a previous shader as an input texture.

For the actual pipeline specification, I could either develop my own simple specification language, or reuse some existing language. Since this project uses WebGL, it was decided to use JavaScript. This had the added bonus of enabling the user to specify a more dynamic pipeline that takes e.g. browser differences into account, with no extra work.

However, the objects used internally by my project are sufficiently complicated that I did not want to expose them directly to the user. Therefore, a simple environment containing methods for creating dummy shaders is created. These dummy shaders can then be converted into the internal format later.

\section{Parameter UI generation}
This stage takes a tree of dummy shaders, and returns the appropriate HTML user interface. For each parameter of the shader which needs specifying - that is, is of type [TODO typeset]uniform and is not already specified - it produces an appropriate set of input elements. 

In the initial implementation, the appropriate shader values would be updated when a new render was requested. However this is inefficient, and makes the UI less responsive. Later implementations therefore use a different method: each input element has associated event listeners that will update the value within the shader, and correctly invalidate the GLOW cache and request the preview be re-rendered.

While simple text boxes provide complete control over the parameters, for certain types of parameter more useful widgets can be provided. The most obvious example is for vec3 and vec4 (vectors of length 3 and for) parameters, which are often used to specify colours. For such parameters, I constructed a colour picking widget based on [REF], as can be seen in [TODO].

\subsection{Struct handling}
Since GLOW expects structure parameters to be specified by ordinary JavaScript objects, which are also used to specify parameters, structs can be easily and simply handled by recursing on the parameter generation function. While for languages like C, this could produce an infinite loop if the structure references itself, GLSL does not permit this. However [TODO test this].

\section{Pipeline generation}
\subsection*{Pipeline linearisation}
The primary task of the pipeline generation stage is converting the shader DAG into a list. This can be acheived using a topological sort. Initially the algorithm decribed by Cormen et al \cite{topsort} [REF 
Cormen, Thomas H.; Leiserson, Charles E.; Rivest, Ronald L.; Stein, Clifford (2001), "Section 22.4: Topological sort", Introduction to Algorithms (2nd ed.), MIT Press and McGraw-Hill, pp. 549–552, ISBN 0-262-03293-7.] was used. The algorithm is shown in \ref{simpletopological}.
\begin{algorithm}
\label{simpletopological}
\begin{algorithmic}
\State $L \gets $ Empty list that will contain the sorted nodes
\State $S \gets $ Set of all nodes with no outgoing edges
\ForAll{node n in S}
    visit(n)
\EndFor 
\Function{visit}{node}
    \If{n has not been visited yet}
        mark n as visited
        \ForAll{node m with an edge from m to n}
            visit(m)
        \EndFor
        add n to L
    \EndIf
\EndFunction
\end{algorithmic}
\end{algorithm}
However, the algorithm shown in \ref{simpletopological} cannot detect when the graph contains a loop. This situation will only occur if the user inputs an invalid shader graph -- in such a case, the program should produce an error.

\subsection*{Pipeline initialization}
In this stage, each shader object in the pipeline must be assigned an actual GLOW shader. This is the object that contains the actual compiled shader that will be called at rendertime. Initially, this was done in the most straightforward manner possible, by simply iterating through the pipeline and generating new GLOW objects after any modification. This was later modified such that changing parameters would not require this -- GLOW objects are created before the Parameter UI Generation stage, with dummy parameter values. 

\section{Rendering}
The rendering process itself is kept as simple, since we try to offload as much work as possible to other stages. This is important for later extensions involving animation, where we want the render loop to run as quickly as possible. The render loop simply iterates through the shader pipeline list, binding the FBO associated with each shader, rendering the shader, and then unbinding the FBO. The FBO binding is skipped for the final shader. While the initial render function cleared the GLOW cache at the start of each call, we can avoid this as discussed under [REF glow cache]

\section{Optimisation}\subsection{GLOW early initialization}
As discussed in [REF UI params, Pipeline init], in the initial implementation all stages following Parameter UI Generation must be called, at the users request, following each change to a parameter. This also includes a step in which every parameter in the UI is updated. This is obviously suboptimal, both since all parameters must be updated and redundant steps be re-run, and since the user is required to request re-rendering, which reduces the level of interactivity.
The solution to this problem is briefly mentioned in [REF pipeline init].

\subsection{Modified shader detection}
The implementation as discussed above performs the entire shader parsing through pipeline initialization process, throwing away all previous data, whenever a shader program is modified. However, since only one shader program can be modified at once, this is very sub-optimal. This is particularly important since we would like to be able to render a new preview of the pipeline output as quickly as possible.

\subsection{Parameter value propagation}
Some of the advantages in interactivity introduced by the optimisations made in [REF msd] will be useless, since parameters specified for the modified shader will be lost, and will need to be re-entered by the user. However we cannot simply reuse the previous shader parameters since these may have changed.

\subsection{GLOW cache}
[TODO: this is broken]

\subsection{FBO reuse}
In the initial implementation, FBOs are simply discarded and new FBOs are allocated after each pipeline change. This is time consuming and could be avoided. However, since FBOs come in different sizes, we cannot naively allocate some 'pool' of available FBOs. This is achieved by simply keeping

\section{User Interface}
\subsection{Editor}
As discussed in [REF], I chose to use the CodeMirror text editor to provide GLSL syntax highlighting to users. While CodeMirror does not provide GLSL highlighting, it does provide a general, 'C-like' highlighting mode, and simple hooks for custom parsers. Initially, I based my syntax highlighting on the 'C-like' mode provided. This was a simple matter of providing the correct keywords. However, this only provides simple highlighting. Given that each shader is parsed anyway, it would seem sensible to modify the parser used for parameter extraction to also provide syntax highlighting information. [TODO: actually do this]

\subsection{Layout}
The initial layout of the UI was straightforward, consisting of a pair of text editors for each shader program, an editor for the pipeline and parameter specification, a sequence of control buttons and a preview box. [TODO picture]. While this layout was spartan and not user friendly, it was sufficient for initial testing.
\subsection{Improved Layout}
The layout described above suffers from a number of usability problems. Firstly, the UI does not fit at all on the screen of the average user. This requires the user to be constantly scrolling to use it. Secondly, the control buttons are unintuitive, and increase the expected length of a user's development cycle. From [TODO another picture] we can see the improved layout. Shader programs occupy the top left side of the screen, and can be switched between by clicking on the shader's name. The right hand side of the screen is occupied by the parameter specification UI at the bottom, and the preview at the top. The control buttons have been eliminated entirely using the hooks developed in [REF optimisation]

\cleardoublepage
\chapter{Evaluation}
\section{Parser correctness}
In order to assist with development of the parser, and in particular to assist in regression testing, a test harness was developed for the testing the correctness of parameters returned by the parser. Since the parser does not access any browser-specific APIs, it was possible to perform this testing in a simple, automated way on the command line. The tests consist of pairs of shaders and an object containing the correct parameters for that shader. In [REF figure], we see a graph of the percentage of the test shaders correctly parsed vs. git revision.

\begin{tikzpicture}
\begin{axis}[xlabel={git revision},ylabel={tests passed}]
\addplot file {parser.data};
\end{axis}
\end{tikzpicture}

\section{Pipeline correctness}

\section{Speed}
\subsubsection{Effect of modified shader detection}
In [REF optimisation], we discussed the modification of the pipeline initialization process to avoid unnecessary shader re-parsing, compilation, and pipeline generation. 

\subsection{Effect of FBO reuse}
Also in [REF optimisation], we discuss the modification of the pipeline initialization process to reuse FBOs where possible. While the main motivation for this is to avoid allocating textures unnecessarily, this has possible speed benefits. [TODO graph]

\section{Texture usage}

\cleardoublepage
\chapter{Conclusion}

\section{Main Results}
The project has been successful. The core of the project has been completed and functions correctly, and with the addition of the various optimisations and extensions, is sufficiently interactive to be useful. I have made the project source code availabe to others via GitHub, and I hope that it will prove useful.

\section{Lessons learned}
The combination of WebGL and Linux is still a little unstable at times, and as such I would be apprehensive about relying so much on a very recent technology for future projects.

While I was initially apprehensive about using JavaScript given it's reputation as an ugly language to work with, I was pleasantly surprised by it's novel object model and quite functional underpinnings.

\section{Further work}
As this project was only partially focused on the development of a user interface, the current UI is fairly basic. Further work could be done to make the UI more usable, and possibly to conduct a user study to test its effectiveness.

Both the specification and implementation of WebGL have changed significantly since their first proposal, and I expect them to continue to do so for some time yet. As such further work may be necessary in the future as these change. Of particular note is the current lack of multiple render targets in current implementations [REF]. If in the future implementations begin to support multiple render targets, adding support for these to my project will be a moderately non-trivial task.

\cleardoublepage

%%%%%%%%%%%%%%%%%%%%%%%%%%%%%%%%%%%%%%%%%%%%%%%%%%%%%%%%%%%%%%%%%%%%%
% the bibliography

\addcontentsline{toc}{chapter}{Bibliography}
\bibliography{refs}
\cleardoublepage

%%%%%%%%%%%%%%%%%%%%%%%%%%%%%%%%%%%%%%%%%%%%%%%%%%%%%%%%%%%%%%%%%%%%%
% the appendices
\appendix

%\chapter{Latex source}

%\section{diss.tex}
%{\scriptsize\verbatiminput{diss.tex}}

%\section{proposal.tex}
%{\scriptsize\verbatiminput{proposal.tex}}

%\section{propbody.tex}
%{\scriptsize\verbatiminput{propbody.tex}}



%\cleardoublepage

%\chapter{Makefile}

%\section{\label{makefile}Makefile}
%{\scriptsize\verbatiminput{makefile.txt}}

%\section{refs.bib}
%{\scriptsize\verbatiminput{refs.bib}}


\cleardoublepage

\chapter{Project Proposal}

%
% Draft #1 (final?)

\vfil

\centerline{\Large Part II Computer Science Project Proposal}
\vspace{0.4in}
\centerline{\Large Shader Compositor }
\vspace{0.4in}
\centerline{\large Joseph Seaton, Fitzwilliam College}
\vspace{0.3in}
\centerline{\large Originator: Christian Richardt}
\vspace{0.3in}
\centerline{\large 21 November 2000}

\vfil

\subsection*{Special Resources Required}

\vspace{0.2in}

\noindent
{\bf Project Supervisor:} Christian Richardt
\vspace{0.2in}

\noindent
{\bf Director of Studies:} Dr R. Harle
\vspace{0.2in}
\noindent
 
\noindent
{\bf Project Overseers:} [TODO]

\vfil
\pagebreak

% Main document

\section*{Introduction}
Designing shaders can be laborious work, involving endless back and forth between tweaking the
code (often minor visual tweaks) and examining the visual output. While programs exist to provide
previews of individual shaders, modern software, especially for computer game engines, often
involves multiple shaders chained together, and little software exists to aid shader writers in testing
such shader pipelines. Furthermore for beginners, the initial process of learning of how to use
shaders is complicated by this extra legwork to compose shaders. The aim of this project is to
alleviate these problems to some degree.

\subsection*{Aside on Shaders}
A 'shader' is a piece of code that runs directly on a GPU. Shaders work on the data flow model, in
that once the code is set up, many data may be fed through it. In OpenGL, which this project will
focus on, such shaders are divided into a number of types. The two types considered here are vertex
shaders and fragment shaders although there are others including geometry shaders which could be
considered as an extension. Vertex shaders operate on a vertex, and its associated data, performing
such operations as transforming a vertex to the screen coordinates. Fragment shaders
(approximately speaking) operate on individual pixels, and are often used for operations such as
texturing a model. Vertex shaders may pass information to fragment shaders, and the underlying
program using the shaders may pass information to both.

\subsection*{The Project}
This project intends to provide an interface to allow a user to create a set of shaders, written in a
slightly annotated version of GLSL. The attributes of these shaders will be detected, along with the
type of each attribute, and, given a specification of how to connect these attributes (provided either
via a GUI or a simple language), a pipeline of these shaders will be constructed.
It should be noted that the construction of said pipeline will require a number of stages, namely
construction of a simple GLSL parser, generation of a DAG of shaders, linearisation of said DAG,
and finally application of this DAG to FrameBuffer Objects for offscreen rendering, finishing in
some rendered model. Since FBOs can be reused, an efficient mapping is non-trivial. Furthermore,
there is room for optimisation of this DAG – consider unification of identical textures, or extraction
of common code.
The interface should also provide some facility to allow the user to easily modify shader parameters
by e.g. selecting textures, or setting the colour of a vector. Addition of simple annotations to the
GLSL may be useful here, such as specifying the range of a vector.
Since this project is intended to be used partly as a learning tool, I propose to use WebGL, a
standard for using OpenGL (including GLSL) on the web, via JavaScript. Since WebGL is cross
platform by design and requires only a recent webbrowser to use, As WebGL is quite a new
technology, it is possible that this may turn out to be infeasible, in which case I would use Python.

\subsection*{Resources Required}
Since this project will involve OpenGL shaders, a recent graphics card supporting a recent version
of OpenGL will be required. To this end I would use my own computer with its AMD HD5450
graphics card. Depending upon the speed at which my test harness runs, I may consider upgrading
to a more powerful card such as the HD6770.

\subsection*{Starting Point}
I have some existing knowledge of OpenGL and GLSL from personal experience, but only to the
extent of small personal projects. I have a reasonable amount of experience with JavaScript having
used it in the past for web development work – but in conjunction with a server-side language.
Substance and Structure of the Project
The project involves writing software providing a user interface in which to input a set of shaders
and the connections between them and view a sample output of the generated shader pipeline on a
sample model.
Most notably to do this the project requires software to be written that given a set of shaders and a
specification of how to connect them, can parse each shader, extract its parameters, and construct
the specified pipeline.
Furthermore the user interface should provide the ability to easily vary the parameters of each
shader on the fly. This and the above comprise the core of the project.
By way of extensions, there is much scope for fleshing out the interface into a fully-featured
development tool, with such additions as example shaders, a more complete set of sample scenes,
and so on. There is also some room for optimisation of the composition process, for example the
unification of identical textures or extraction of common code between shaders.
For evaluation, in order to demonstrate correct composition, a testing framework will be written that
takes a set of semantically obvious or simple shaders and composites them in arbitrary ways and
tests whether the output matches the expected result. Also pipeline speed/memory use comparisons
of before/after the implementation of the extensions will be performed, along with comparisons of
recomplilation speed before/after optimisation (i.e. detection of modified shader). Further
comparisons include comparisons of a generated pipeline against a hand-coded shader, or a
comparison of the time complexity of a given pipeline to a JavaScript program carrying out the
same task.


\section*{Success Criteria}
The following should be achieved:
\begin{enumerate}
\item Interface to construct shaders, specify connections between shaders
\item Detection of parameters of set of shaders, parameters presented to the user somehow
\item Sample scene output for single shader
\item Demonstration of correct composition (with sample scene output) for various shaders with
pipeline specifications
\item Provide basic syntax highlighting of GLSL in interface
\end{enumerate}
Extensions:
Demonstration of some optimisations:
\begin{enumerate}
\item Reuse of FBOs in shader pipeline
\item Automatic unification of identical textures
\item Splitting shared code off into separate shaders
\item Avoiding recompiling shaders unnecessarily
\end{enumerate}
Demonstration of some additional GUI work:
\begin{enumerate}
\item Presentation of sliders with range detected from range annotations in shaders,
\item Some facility for animation (e.g. parameters that are automatically incremented)
\item Shader and/or scene test suite
\end{enumerate}

\section{Timetable and Milestones}
\subsection{Weeks 0-2: 22nd October – 11th November}
Initial reading. In particular, the OpenGL 'red book'. Setup of barebones GUI, associated libraries.
Preparation of automatic backups.
Milestone: dummy interface for single shader.
\subsection{Weeks 3-5: 12th November – 2nd December}
Construction and testing of GLSL parser. Further familiarisation with GLSL.
Milestone: ability to parse simple GLSL
\subsection{Weeks 6-8: 3rd December – 23rd December}
Write shader DAG generating code, DAG linearisation code. Initial work on application to FBOs.
Milestone: generation of linearised shader DAGs
\subsection{Weeks 9-11: 24th December – 13th January}
Complete actual pipeline – given some DAG, the project should be able to render the pipeline to a
quad. Complete interface such that the pipeline order can be specified, either via a GUI or via some
written specification. Basic parser may be sensible.
Milestone: ability to render a given shader pipeline
\subsection{Weeks 12-14: 14th January – 3rd Feburary}
Preparation of progress report. Presentation of shader parameters in GUI. Initial
work on texture unification and other optimisations. Initial work on test suite.
Milestone: Progress report prepared. Ability to vary shader parameters from the GUI. Ability to
unify some identical textures (or reduced FBO or code usage, depending on extension). Some tests
written.
\subsection{Weeks 15-17: 4th February – 24th February}
Presentation of progress report. Further testing of core functionality, expansion of test cases. Bug
fixing. Possibly initial
optimisation work.
Milestone: Progress report given. Large test suite. Lack of obvious bugs in core.
\subsection{Weeks 18-20: 25th February – 16th March}
Additional GUI work and/or optimisation. Slack time for unexpected problems. Optimisations can
be skipped if more time is needed for core work.
Milestone: Better featured GUI and/or some further optimisations.
\subsection{Weeks 21-23: 17th March – 6th April}
Begin work on dissertation. Programming work should be finished by now, although some
expansion of the test suite may be acceptable.
Milestone: complete initial sections of dissertation.
\subsection{Weeks 24-26: 7th April – 27th April}
Finish first draft of dissertation.
\subsection{Weeks 27-29: 28th April – 18th May}
Proofreading and submission.


\end{document}
