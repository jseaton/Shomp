
% Draft #1 (final?)

\vfil

\centerline{\Large Part II Computer Science Project Proposal}
\vspace{0.4in}
\centerline{\Large Shader Compositor }
\vspace{0.4in}
\centerline{\large Joseph Seaton, Fitzwilliam College}
\vspace{0.3in}
\centerline{\large Originator: Christian Richardt}
\vspace{0.3in}
\centerline{\large 21 November 2011}

\vfil

\subsection*{Special Resources Required}
None
\vspace{0.2in}

\noindent
{\bf Project Supervisor:} Christian Richardt
\vspace{0.2in}

\noindent
{\bf Director of Studies:} Dr R. Harle
\vspace{0.2in}
\noindent
 
\noindent
{\bf Project Overseers:} Dr S. Holden, Dr D. J. Greaves

\vfil
\pagebreak

% Main document

\section*{Introduction}
Designing shaders can be laborious work, involving endless back and forth between tweaking the
code (often minor visual tweaks) and examining the visual output. While programs exist to provide
previews of individual shaders, modern software, especially for computer game engines, often
involves multiple shaders chained together, and little software exists to aid shader writers in testing
such shader pipelines. Furthermore for beginners, the initial process of learning of how to use
shaders is complicated by this extra legwork to compose shaders. The aim of this project is to
alleviate these problems to some degree.

\subsection*{Aside on Shaders}
A 'shader' is a piece of code that runs directly on a GPU. Shaders work on the data flow model, in
that once the code is set up, many data may be fed through it. In OpenGL, which this project will
focus on, such shaders are divided into a number of types. The two types considered here are vertex
shaders and fragment shaders although there are others including geometry shaders which could be
considered as an extension. Vertex shaders operate on a vertex, and its associated data, performing
such operations as transforming a vertex to the screen coordinates. Fragment shaders
(approximately speaking) operate on individual pixels, and are often used for operations such as
texturing a model. Vertex shaders may pass information to fragment shaders, and the underlying
program using the shaders may pass information to both.

\subsection*{The Project}
This project intends to provide an interface to allow a user to create a set of shaders, written in a
slightly annotated version of GLSL. The attributes of these shaders will be detected, along with the
type of each attribute, and, given a specification of how to connect these attributes (provided either
via a GUI or a simple language), a pipeline of these shaders will be constructed.
It should be noted that the construction of said pipeline will require a number of stages, namely
construction of a simple GLSL parser, generation of a DAG of shaders, linearisation of said DAG,
and finally application of this DAG to FrameBuffer Objects for offscreen rendering, finishing in
some rendered model. Since FBOs can be reused, an efficient mapping is non-trivial. Furthermore,
there is room for optimisation of this DAG – consider unification of identical textures, or extraction
of common code.
The interface should also provide some facility to allow the user to easily modify shader parameters
by e.g. selecting textures, or setting the colour of a vector. Addition of simple annotations to the
GLSL may be useful here, such as specifying the range of a vector.
Since this project is intended to be used partly as a learning tool, I propose to use WebGL, a
standard for using OpenGL (including GLSL) on the web, via JavaScript. Since WebGL is cross
platform by design and requires only a recent webbrowser to use, As WebGL is quite a new
technology, it is possible that this may turn out to be infeasible, in which case I would use Python.

\subsection*{Resources Required}
Since this project will involve OpenGL shaders, a recent graphics card supporting a recent version
of OpenGL will be required. To this end I would use my own computer with its AMD HD5450
graphics card. Depending upon the speed at which my test harness runs, I may consider upgrading
to a more powerful card such as the HD6770.

\subsection*{Starting Point}
I have some existing knowledge of OpenGL and GLSL from personal experience, but only to the
extent of small personal projects. I have a reasonable amount of experience with JavaScript having
used it in the past for web development work – but in conjunction with a server-side language.
Substance and Structure of the Project
The project involves writing software providing a user interface in which to input a set of shaders
and the connections between them and view a sample output of the generated shader pipeline on a
sample model.
Most notably to do this the project requires software to be written that given a set of shaders and a
specification of how to connect them, can parse each shader, extract its parameters, and construct
the specified pipeline.
Furthermore the user interface should provide the ability to easily vary the parameters of each
shader on the fly. This and the above comprise the core of the project.
By way of extensions, there is much scope for fleshing out the interface into a fully-featured
development tool, with such additions as example shaders, a more complete set of sample scenes,
and so on. There is also some room for optimisation of the composition process, for example the
unification of identical textures or extraction of common code between shaders.
For evaluation, in order to demonstrate correct composition, a testing framework will be written that
takes a set of semantically obvious or simple shaders and composites them in arbitrary ways and
tests whether the output matches the expected result. Also pipeline speed/memory use comparisons
of before/after the implementation of the extensions will be performed, along with comparisons of
recomplilation speed before/after optimisation (i.e. detection of modified shader). Further
comparisons include comparisons of a generated pipeline against a hand-coded shader, or a
comparison of the time complexity of a given pipeline to a JavaScript program carrying out the
same task.


\section*{Success Criteria}
The following should be achieved:
\begin{enumerate}
\item Interface to construct shaders, specify connections between shaders
\item Detection of parameters of set of shaders, parameters presented to the user somehow
\item Sample scene output for single shader
\item Demonstration of correct composition (with sample scene output) for various shaders with
pipeline specifications
\item Provide basic syntax highlighting of GLSL in interface
\end{enumerate}
Extensions:
Demonstration of some optimisations:
\begin{enumerate}
\item Reuse of FBOs in shader pipeline
\item Automatic unification of identical textures
\item Splitting shared code off into separate shaders
\item Avoiding recompiling shaders unnecessarily
\end{enumerate}
Demonstration of some additional GUI work:
\begin{enumerate}
\item Presentation of sliders with range detected from range annotations in shaders,
\item Some facility for animation (e.g. parameters that are automatically incremented)
\item Shader and/or scene test suite
\end{enumerate}

\section{Timetable and Milestones}
\subsection{Weeks 0--2: 22nd October –- 11th November}
Initial reading. In particular, the OpenGL 'red book'. Setup of barebones GUI, associated libraries.
Preparation of automatic backups.
Milestone: dummy interface for single shader.
\subsection{Weeks 3--5: 12th November -– 2nd December}
Construction and testing of GLSL parser. Further familiarisation with GLSL.
Milestone: ability to parse simple GLSL
\subsection{Weeks 6--8: 3rd December -– 23rd December}
Write shader DAG generating code, DAG linearisation code. Initial work on application to FBOs.
Milestone: generation of linearised shader DAGs
\subsection{Weeks 9--11: 24th December -– 13th January}
Complete actual pipeline – given some DAG, the project should be able to render the pipeline to a
quad. Complete interface such that the pipeline order can be specified, either via a GUI or via some
written specification. Basic parser may be sensible.
Milestone: ability to render a given shader pipeline
\subsection{Weeks 12--14: 14th January -– 3rd Feburary}
Preparation of progress report. Presentation of shader parameters in GUI. Initial
work on texture unification and other optimisations. Initial work on test suite.
Milestone: Progress report prepared. Ability to vary shader parameters from the GUI. Ability to
unify some identical textures (or reduced FBO or code usage, depending on extension). Some tests
written.
\subsection{Weeks 15--17: 4th February -– 24th February}
Presentation of progress report. Further testing of core functionality, expansion of test cases. Bug
fixing. Possibly initial
optimisation work.
Milestone: Progress report given. Large test suite. Lack of obvious bugs in core.
\subsection{Weeks 18--20: 25th February -– 16th March}
Additional GUI work and/or optimisation. Slack time for unexpected problems. Optimisations can
be skipped if more time is needed for core work.
Milestone: Better featured GUI and/or some further optimisations.
\subsection{Weeks 21--23: 17th March -– 6th April}
Begin work on dissertation. Programming work should be finished by now, although some
expansion of the test suite may be acceptable.
Milestone: complete initial sections of dissertation.
\subsection{Weeks 24--26: 7th April -– 27th April}
Finish first draft of dissertation.
\subsection{Weeks 27--29: 28th April -– 18th May}
Proofreading and submission.
